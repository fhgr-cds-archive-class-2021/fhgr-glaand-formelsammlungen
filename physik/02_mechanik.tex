\noindent
\paragraph{Mechanik}\mbox{}\\
\begin{tabularx}{\columnwidth}{@{}XXX@{}}
	Impuls                     & $ \vec{p} = m\cdot\vec{v} $                & $ \lbrack{p}\rbrack = kg\cdot \frac{m}{s} = N\cdot s$      \\ \hline
	Impulserhaltung            & $ \vec{p}_{tot} = \sum \vec{p}_i = konst $ & $\vec{p}_{tot}$: Gesamtimpuls, $\vec{p}_{i}$: Einzelimpuls \\ \hline
	Impulserhaltungssatz (IES) & $\sum P_E = \sum P_A$                      & $P_{E1} + P_{E2} + ... = P_{A1} + P_{A2} + ...$            \\ \hline
\end{tabularx}
\noindent
\begin{tabularx}{\columnwidth}{@{}Xl@{}}
	IES-Stossprozess                            & $ m_{1} \cdot v_{E1} + m_{2} \cdot v_{E2} = \underbrace{m_{1} \cdot v_{A1} + m_{2} \cdot v_{A2}}_\text{= 0} $ \\ \hline
	IES-Kollisionsprozess                       & $ \left(m_{1} + m_{2}\right) \cdot v_{E} = m_{1} \cdot v_{A1} + m_{2} \cdot v_{A2} $                          \\ \hline
	IES-Abwurfprozess (Gleicher Inertialsystem) & $ m_{1} \cdot v_{E1} + m_{2} \cdot v_{E2} = \underbrace{m_{1} \cdot v_{A1} + m_{2} \cdot v_{A2}}_\text{= 0} $ \\ \hline
	Theorie verti. Wurfes                       & $ \tilde{v}_E = \sqrt{2g\Delta h} $                                                                           \\ \hline
	Newton 1. Axiom                             & Trägheitsprinzip ($ F = 0 $)                                                                                  \\ \hline
	Newton 2. Axiom                             & Aktionsprinzip ($ F_{res} = m\cdot a $)                                                                       \\ \hline
	Newton 3. Axiom                             & Reaktionsprinzip ($ \tilde{F}_{res} = -F_{res} $)                                                             \\ \hline
	Gleichmässig beschleunigte Bewegung         & $ \vec{v} = \vec{v}_0 + \vec{a}t $                                                                            \\ \hline
	Gleichmässig beschleunigte Bewegung         & $ \vec{r} = \vec{r}_0 + \vec{v}_0t + \frac{1}{2}\vec{a}t^2 $                                                  \\ \hline
	Gleichmässig beschleunigte Bewegung         & $ \vec{a} = \vec{v}\,' = \vec{r}\,'' $                                                                        \\ \hline
\end{tabularx}
\noindent
\begin{tabularx}{\columnwidth}{@{}XXX@{}}
	Geschwindigkeit                           & $ v = \frac{\Delta s}{\Delta t} = \frac{s_E - s_A}{t_E - t_A} $  & $ [v] = \frac{m}{s} $                                                                                                              \\ \hline
	Geschwindigkeit                           & $v^2 = v^2_0 + 2a(s-s_0)$                                        & $v^2 = v^2_0 + 2a\cdot \Delta s$                                                                                                   \\ \hline
	Beschleunigung                            & $ a = \frac{\Delta v}{\Delta t} = \frac{v_E - v_A}{t_E - t_A} $  & $ [a] = \frac{m}{s^2} $                                                                                                            \\ \hline
	Beschleunigung                            & $ a = \frac{{v_E}^2}{2\cdot\Delta s} $                           & $ [a] = \frac{m}{s^2} $                                                                                                            \\ \hline
	Gravitationskraft                         & $ F_g = m\cdot a_g $                                             & $ [N] = \frac{kg\cdot m}{s^2} $                                                                                                    \\ \hline
	Result. Kraft                             & $ F_{res} = m\cdot a $                                           & $ [N] = \frac{kg\cdot m}{s^2} $                                                                                                    \\ \hline
	Kraft via Impuls                          & $ \frac{\Delta p}{\Delta t} = \frac{v\cdot \Delta m}{\Delta t} $                                                                                                                                      \\ \hline
	Kraft via Impuls                          & $\vec{F}_{res} = \vec{p}\,'$                                                                                                                                                                          \\ \hline
	Kraftstoss                                & $ \vec{F}\Delta t = \Delta\vec{p} $                              & Falls $\vec{F}$ konst.                                                                                                             \\ \hline
	Kraftstoff (Integral)                     & $ \int_{t_1}^{t_2} \vec{F}dt \,dx $                              & Falls $\vec{F}$ variiert.                                                                                                          \\ \hline
	Gleitreibungskraft                        & $ F_{GR} = \mu_G\cdot F_N $                                                                                                                                                                           \\ \hline
	Haftreibungskraft                         & $ F_{HR} = \mu_H\cdot F_N $                                                                                                                                                                           \\ \hline
	Rollwiderstandskraft                      & $ F_{RR} = \mu_R\cdot F_N $                                                                                                                                                                           \\ \hline
	Hangabtriebskraft                         & $ F_H = F_G \cdot sin(\alpha) $                                  & Schiefe Ebene                                                                                                                      \\ \hline
	Hangabtriebskraft                         & $ F_H = F_N \cdot tan(\alpha) $                                  & Schiefe Ebene                                                                                                                      \\ \hline
	Normalkraft                               & $ F_N = F_G \cdot cos(\alpha) $                                  & Schiefe Ebene                                                                                                                      \\ \hline
	Normalkraft                               & $ F_N = \frac{F_H}{tan(\alpha)} $                                & Schiefe Ebene                                                                                                                      \\ \hline
	Steigung S                                & $S = \frac{h}{b}$                                                & auf b = 100m bezogen und mist in \% ausgedruckt                                                                                    \\ \hline
	Steig S - $cos(\alpha)$                   & $cos(\alpha) = \frac{1}{\sqrt{S^2+1}}$                                                                                                                                                                \\ \hline
	Steig S - $sin(\alpha)$                   & $sin(\alpha) = \frac{\sqrt{S^2+1}}{S}$                                                                                                                                                                \\ \hline
	Federkraft\linebreak (HOOKE'sches Gesetz) & $ F_F = D\cdot y $                                               & $y$ = $|l - l_o|$\linebreak $l_0$: entspannte Feder\linebreak l: gespannte Feder\linebreak $D$: Federkonstante $[D] = \frac{N}{m}$ \\ \hline
	Dichte                                    & $\varrho = \frac{m}{V}$                                          & $[\varrho] = \frac{kg}{m^3}$                                                                                                       \\ \hline
	max. dichte Wasser                        & 4°C                                                                                                                                                                                                   \\ \hline
	Strömungswiderstand                       & $ F_w = c_w \cdot \frac{\varrho v^2}{2}\cdot A $                 & $c_w: Widerstandszahl$                                                                                                             \\ \hline
	Arbeit (Konstanter Kraft)                 & $ W = F \cdot \Delta s$                                          & $[W] = N\cdot m = W\cdot s = J$                                                                                                    \\ \hline
	Arbeit (Varriierte Kraft)                 & $ W=\int_{s_A}^{s_B} F_sds \,dx  $                                                                                                                                                                    \\ \hline
	Energie                                   & $E$                                                              & $[E] = J = W\cdot s$                                                                                                               \\ \hline
	Rel. Arbeit und Energie                   & $\Delta E = W_{AB}$                                                                                                                                                                                   \\ \hline
	kin. Energie                              & $E_k = \frac{1}{2}mv^2$                                                                                                                                                                               \\ \hline
	pot. Energie                              & $E_p = mgh$                                                                                                                                                                                           \\ \hline
	Spannenergie Feder                        & $E_f = \frac{1}{2}Dy^2$                                                                                                                                                                               \\ \hline
\end{tabularx}
\noindent
\begin{tabularx}{\columnwidth}{@{}XXX@{}}
	Energieerhaltung            & $ \vec{E}_{tot} = \sum \vec{E}_i = konst $                           & $\vec{E}_{tot}$: Gesamtenergie, $\vec{E}_{i}$: Teilenergie \\ \hline
	Energieerhaltungssatz (EES) & $\sum E_E = \sum E_A$                                                & $E_{E1} + E_{E2} + ... = E_{A1} + E_{A2} + ...$            \\ \hline
	EES pot. \& kin.            & $E_{pE} + E_{kinE} =$                                                & $E_{pA} + E_{kinA}$                                        \\ \hline
	Wirkungsgrad                & $\mu = \frac{W_2}{W_1} = \frac{P_2}{P_1}$                            & $W_1,P_1: aufgeno.\linebreak W_2,P_2: nutzbare$            \\ \hline
	Wirkungsgrad                & zwischen mehrere Maschinen                                           & $\mu_1 \cdot \mu_2 \cdot ...$                              \\ \hline
	1 PS                        & $\approx 735.499 W$                                                                                                               \\ \hline
	Leistung                    & $P$                                                                  & $[P] = \frac{J}{s} = W\,(Watt)$                            \\ \hline
	mittlere Leistung           & $\overline{P} = \frac{W_{AB}}{\Delta t} = \frac{\Delta E}{\Delta t}$                                                              \\ \hline
	momentane Leistung          & $P = \frac{dW}{dt} = \vec{F}\cdot\vec{v}$                                                                                         \\ \hline
\end{tabularx}
\vspace{1mm}

\paragraph{Druck}\mbox{}\\
\noindent
\begin{tabularx}{\columnwidth}{@{}XXX@{}}
	Druck                      & $p = \frac{F_N}{A}$                                & $[p] = \frac{N}{m^2} = Pa$ \\ \hline
	Druck                      & $p = \frac{\Delta F_N}{\Delta A}$                  & $[p] = \frac{N}{m^2} = Pa$ \\ \hline
	Schweredruck               & $\Delta p = \varrho gh$                                                         \\ \hline
	Barometrische Höhenformel  & $p = p_0 \cdot e^{-\frac{\varrho_0g}{p_0}\cdot h}$                              \\ \hline
	Hydraulik                  & $\frac{F_d}{A_d} = p = \frac{F_p}{A_p}$                                         \\ \hline
	verall. Expo Druck         & $p(h) = p_0 a^{\frac{h-h_0}{\Sigma}}$                                           \\ \hline
	1 Torr                     & $\frac{101'325}{760} Pa$                                                        \\ \hline
	Gas-Druck \& Quecksilber   & $p = \varrho_{Hg} \cdot g \cdot \Delta h$                                       \\ \hline
	1 mm Hg                    & $= \varrho_{Hg} \cdot g \cdot 1mm$                 & $\approx 1.33 Pa$          \\ \hline
	1 bar                      & $1.0\cdot 10^5 Pa$                                                              \\ \hline
	Verhältnis Torr \& 1 mm Hg & $\frac{1 Torr}{1 mm Hg} \approx 1.00$                                           \\ \hline
\end{tabularx}

\paragraph{Temperatur}\mbox{}\\
\noindent
\begin{tabularx}{\columnwidth}{@{}XXX@{}}
	Absoluter Nullpunkt                    & 0K = -273.15°C                                                                                              \\ \hline
	absolute Temperatur                    & Kelvin                                                                                                      \\ \hline
	Temperaturdifferenz                    & $\Delta T = \Delta \vartheta$                                                                               \\ \hline
	Griechische Zeichen                    & $\vartheta$ für Celsius                              & $T$ für Kelvin                                       \\ \hline
	Längenausdehnung                       & $\Delta l \approx \alpha l \Delta T$                                                                        \\ \hline
	Volumenausdehnung                      & $\Delta V \approx \gamma V \Delta T$                                                                        \\ \hline
	Isotrope Festkörper                    & $\gamma \approx 3\alpha$                                                                                    \\ \hline
	Wärmemenge                             & $Q$                                                  & $[Q] = J$                                            \\ \hline
	Innere Energie                         & $U$                                                  & $[U] = J$                                            \\ \hline
	Spez. Wärmekapazität c                 & $\Delta Q = cm\Delta T$                              & $[c] = \frac{J}{(kg\cdot K)}$                        \\ \hline
	Molare Wärmekapazität C                & $\Delta Q = nC\Delta T$                              & $[C] = \frac{J}{(mol\cdot K)}$                       \\ \hline
	EES-Wärmemenge                         & $\Delta Q_1 + \Delta Q_2 + \Delta Q_3$               & $= 0$                                                \\ \hline
	Stoffmenge                             & $n = \frac{m}{M}$                                    & $[n] = mol$; M: Molare Masse; $[M] = \frac{kg}{mol}$ \\ \hline
	Schmelz-Erstarrungswärme $L_f$         & $Q = L_f\cdot m$                                     & $[L_f] = \frac{J}{kg}$                               \\ \hline
	Verdampguns-Kondensswärme $L_v$        & $Q = L_v\cdot m$                                     & $[L_v] = \frac{J}{kg}$                               \\ \hline
	1. Hauptsatz Thermodynamik             & $\Delta U = Q + W$                                                                                          \\ \hline
	U                                      & $= \frac{1}{2}f\cdot N\cdot k_B \cdot T$                                                                    \\ \hline
	U                                      & $= \frac{1}{2}f\cdot n\cdot R \cdot T$                                                                      \\ \hline
	U                                      & $= \frac{1}{2}f\cdot p \cdot V$                                                                             \\ \hline
	ideale Gase $C_v$                      & $C_v = \frac{1}{2}\cdot f \cdot R$                   & $Q = nC_v\Delta T$                                   \\ \hline
	Festkörper $C_p$                       & $C_p = 3\cdot R$                                                                                            \\ \hline
	Molare Wärmekapazität der idealen Gase & $C_p - C_v = R$                                                                                             \\ \hline
	Wärmekapazität eines Materials         & $c^{*} = m \cdot c = n \cdot C$                                                                             \\ \hline
	Thermische Zustandsgleichung           & $pV = NkT = nRT$                                                                                            \\ \hline
	$R = N_A \cdot k_B $                   & $N = n \cdot N_A$                                                                                           \\ \hline
	Druck des ideales Gases                & $p = \frac{1}{3}\cdot\frac{N}{V}m\overline{v^2}$     & $= \frac{2}{3}\cdot\frac{N}{V}\cdot\overline{E}_k$   \\ \hline
	Mittlere Translationsenergie           & $\overline{E}_k = \frac{1}{2}\cdot m \overline{v^2}$ & $= \frac{3}{2}\cdot k_B \cdot T$                     \\ \hline
	Mittlere quad. Geschwindigkeit         & $V_{rms} = \sqrt{\overline{v^2}}$                    & $= \sqrt{\frac{3kT}{m}}$                             \\ \hline
	1 : 2 : 3                              & $\frac{1}{6}$ : $\frac{2}{6}$ : $\frac{3}{6}$                                                               \\ \hline
\end{tabularx}

\noindent
\begin{tabularx}{\columnwidth}{@{}XXX@{}}
	Isotherm          & $Boyle-Mariotte\,Gesetz$                 & $(p\cdot V = konst.)$    \\ \hline
	Isobar            & $Gay-Lussac\,Gesetz$                     & $(\frac{V}{T} = konst.)$ \\ \hline
	Isochor           & $Amontons\,Gesetz$                       & $(\frac{p}{T} = konst.)$ \\ \hline
	Boyle-Mariotte    & $p_1\cdot V_1 = p_2\cdot V_2$                                       \\ \hline
	Gay-Lussac        & $\frac{V_E}{T_E} = \frac{V_A}{T_A}$                                 \\ \hline
	Amontons          & $\frac{p_E}{T_E} = \frac{p_A}{T_A}$                                 \\ \hline
	Luftblase Taucher & $p_0 = p_L + \varrho \cdot  g \cdot h_0$                            \\ \hline
	Druckintern       & $p = p_l + \Delta p$                                                \\ \hline
\end{tabularx}
\vspace{1mm}

\paragraph{Freiheitsgrade}\mbox{}\\
\noindent
\begin{itemize}
	\item Freiheitsgrade der Translation: $f_T$
	\item Freiheitsgrade der Rotation: $f_R$
	\item Freiheitsgrade der Schwingung (Wird doppelt gezählt): $f_S$
	\item $f = f_T + f_R + 2\cdot f_s$
\end{itemize}

\noindent
\begin{tabularx}{\columnwidth}{@{}XXXXX@{}}
	\hline
	                              & $f_T$ & $f_R$ & $f_S$  & Beispiel       \\ \hline
	1-atomig                      & 3     & 0     & 0      & He, Ar, Ne     \\ \hline
	2-atomig                      & 3     & 2     & 1      & ${H_2}, {O_2}$ \\ \hline
	3- und mehr-atomig, linear    & 3     & 2     & $3n-5$ & $CO_2$         \\ \hline
	3- und mehr-atomig, gewinkelt & 3     & 3     & $3n-6$ & $H_2O$, $CH_4$ \\ \hline
\end{tabularx}
\linebreak
$n$ = Anzahl atome
\vspace{1mm}

\paragraph{Drehbewegung}\mbox{}\\
\noindent
\begin{tabularx}{\columnwidth}{@{}XXX@{}}
	Umlaufzeit                       & $T$                                                                  & $[T] = s$                              \\ \hline
	Umdrehungen						 & $\frac{\varphi}{2*\pi}$ \\ \hline
	Frequenz                         & $f=\frac{1}{T}$                                                      & $[f] = s^{-1} = Hz$                    \\ \hline
	Winkelkoordinate                 & $\varphi = \frac{b}{r}$                                              & $[\varphi] = rad = \frac{m}{m}$        \\ \hline
	Drehwinkel                       & $\varphi = \varphi_0 + \omega_0 t + \frac{1}{2}\alpha t^2$                                                    \\ \hline
	Winkelgeschwindigkeit            & $\omega = \frac{\Delta \varphi}{\Delta t} = \frac{2\pi}{T} = 2\pi f$ & $[\omega] = \frac{rad}{s} = s^{-1}$    \\ \hline
	Winkelgeschwindigkeit            & $\omega = \frac{d\varphi}{dt} = \varphi\,'$                                                                   \\ \hline
	Winkelgeschwindigkeit            & $\omega = \omega_0 + \alpha t$                                                                                \\ \hline
	Winkelgeschwindigkeit            & $\omega^2 = \omega_0^2 + 2\alpha\cdot $                              & $\left(\varphi-\varphi_{0}\right)$     \\ \hline
	Winkelgeschwindigkeit            & $\omega^2 = \omega_0^2 + 2\alpha\cdot\Delta \varphi$                                                          \\ \hline
	Bahngeschwindigkeit              & $v = r\cdot \omega$                                                                                           \\ \hline
	Bahngeschwindigkeit              & $\vec{v}_i = \vec{w} \times \vec{r}_i$                                                                        \\ \hline
	Bahngeschwindigkeit              & $|\vec{v}_i| = |\vec{w}| \cdot |\vec{r}_i| \cdot sin(\varphi)$                                                \\ \hline
	Winkelbeschleunigung             & $ \alpha = \frac{d\omega}{dt} = \omega\,' = \varphi\,''$                                                      \\ \hline
	Bahnbeschleunigung eines Punktes & $a_i = r_i\cdot \alpha$                                                                                       \\ \hline
	Zentripetal-beschleunigung       & $a_z = \frac{v^2}{r} = r\cdot \omega^2$                                                                       \\ \hline
	$\vec{F}_ZP$                     & $= m\cdot \frac{v^2}{r}$                                             & $= m\cdot \omega^2\cdot r$             \\ \hline
	Zentripetalkraft                 & $F_{ZP}$                                                             & $F_{ZP} = F_{res} = F_g + F_{seil}$    \\\hline
	Zentrifugalkraft                 & negatives von Zentripetalkraft                                                                                \\ \hline
	Kraft-Dreieck                    & $cot(\eta) = \frac{F_g}{F_{ZP}}$                                     & $cos(\eta) = \frac{g}{l\cdot\omega^2}$ \\ \hline
	Trägheitsmoment I                & $I = \sum_{i} m_i\cdot r_i^2$                                                                                 \\ \hline
	I                                & $= \int r^2dm$                                                       & $= m_0\cdot r_R^2$                     \\ \hline
	Steiner'scher Satz               & $I = I_S + mh^2$                                                                                              \\ \hline
	2. Netwon Axiom                  & $M_{ext} = I \cdot \alpha = \frac{dL}{dt}$                                                                    \\ \hline
\end{tabularx}
\noindent
\begin{tabularx}{\columnwidth}{@{}XXX@{}}
	Tangentialkomponente von $\vec{F}$ & $F_t = ma_t$                                                                     \\ \hline
	Drehmoment einer Achse             & $M = r\cdot F_t$                              & $[M] = N\cdot m$                 \\ \hline
	Summation Drehmoment               & $\sum_{i} M_i = I\cdot\alpha$                                                    \\ \hline
	Drehmoment Punktes                 & $\vec{M} = \vec{r} \times \vec{F}$                                               \\ \hline
	Drehmoment Vorzeichen              & Gegenuhrzeiger ist Positiv                    & Uhrzeiger ist negativ            \\ \hline
	Drehimpulses                       & $\vec{L} = \vec{r} \times \vec{p}$                                               \\ \hline
	$\vec{L}\,||\,\vec{\omega}$        & Nur falls Punkt nicht auf z-achse             & $\vec{L} = I \cdot \vec{\omega}$ \\ \hline
	Arbeit                             & $W = M \cdot \varphi$                                                            \\ \hline
	Kin. Energie                       & $E_{kin} = \frac{1}{2}\cdot I \cdot \omega^2$                                    \\ \hline
	Leistung                           & $P = M \cdot \omega$                                                             \\ \hline
	2. Netwon Axiom                    & $M_{ext} = I \cdot \alpha = \frac{dL}{dt}$                                      \\ \hline
	Drehimpulserhaltung				   & $L_1 = L_0 + \tilde{L_0}$ \\ \hline
	Drehimpulses                       & $\vec{L_1} = I_1*\omega_1$                                               \\ \hline
	Impuls							   & $p = r*m*v_0$ \\ \hline
	Drehimpulserh. formuliert		   & $I_1*\omega_1 = 0 + r*p$ \\ \hline
	Drehimpuserh. ratio 			   & $\frac{\omega_2}{\omega_1} = \frac{I_1}{I_2}$ \\ \hline

\end{tabularx}