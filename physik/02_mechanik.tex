\noindent
\paragraph{Mechanik}\mbox{}\\
\begin{tabularx}{\columnwidth}{@{}XXX@{}}
	Impuls                     & $ \vec{p} = m\cdot\vec{v} $                & $ \lbrack{p}\rbrack = kg\cdot \frac{m}{s} = N\cdot s$      \\ \hline
	Impulserhaltung            & $ \vec{p}_{tot} = \sum \vec{p}_i = konst $ & $\vec{p}_{tot}$: Gesamtimpuls, $\vec{p}_{i}$: Einzelimpuls \\ \hline
	Impulserhaltungssatz (IES) & $\sum P_E = \sum P_A$                      & $P_{E1} + P_{E2} + ... = P_{A1} + P_{A2} + ...$            \\ \hline
\end{tabularx}
\begin{tabularx}{\columnwidth}{@{}Xl@{}}
	IES-Stossprozess                            & $ m_{1} \cdot v_{E1} + m_{2} \cdot v_{E2} = \underbrace{m_{1} \cdot v_{A1} + m_{2} \cdot v_{A2}}_\text{= 0} $ \\ \hline
	IES-Kollisionsprozess                       & $ \left(m_{1} + m_{2}\right) \cdot v_{E} = m_{1} \cdot v_{A1} + m_{2} \cdot v_{A2} $                          \\ \hline
	IES-Abwurfprozess (Gleicher Inertialsystem) & $ m_{1} \cdot v_{E1} + m_{2} \cdot v_{E2} = \underbrace{m_{1} \cdot v_{A1} + m_{2} \cdot v_{A2}}_\text{= 0} $ \\ \hline
	Theorie verti. Wurfes                       & $ \tilde{v}_E = \sqrt{2g\Delta h} $                                                                           \\ \hline
	Newton 1. Axiom                             & Trägheitsprinzip ($ F = 0 $)                                                                                  \\ \hline
	Newton 2. Axiom                             & Aktionsprinzip ($ F_{res} = m\cdot a $)                                                                       \\ \hline
	Newton 3. Axiom                             & Reaktionsprinzip ($ \tilde{F}_{res} = -F_{res} $)\\ \hline
	Gleichmässig beschleunigte Bewegung & $ \vec{v} = \vec{v}_0 + \vec{a}t $ \\ \hline                                                              
	Gleichmässig beschleunigte Bewegung & $ \vec{r} = \vec{r}_0 + \vec{v}_0t + \frac{1}{2}\vec{a}t^2 $ \\ \hline                                                              
	Gleichmässig beschleunigte Bewegung & $ \vec{a} = \vec{v}\,' = \vec{r}\,'' $ \\ \hline                                                              
\end{tabularx}
\begin{tabularx}{\columnwidth}{@{}XXX@{}}
	Geschwindigkeit   & $ v = \frac{\Delta s}{\Delta t} = \frac{s_E - s_A}{t_E - t_A} $ & $ [v] = \frac{m}{s} $         \\ \hline
	Beschleunigung    & $ a = \frac{\Delta v}{\Delta t} = \frac{v_E - v_A}{t_E - t_A} $ & $ [a] = \frac{m}{s^2} $       \\ \hline
	Beschleunigung    & $ a = \frac{{v_E}^2}{2\cdot\Delta s} $ & $ [a] = \frac{m}{s^2} $ \\ \hline
	Gravitationskraft & $ F_g = m\cdot a_g $                                            & $ [N] = \frac{kg\cdot m}{s} $ \\ \hline
	Result. Kraft     & $ F_{res} = m\cdot a $                                          & $ [N] = \frac{kg\cdot m}{s} $ \\ \hline
	Kraft via Impuls  & $ \frac{\Delta p}{\Delta t} = \frac{v\cdot \Delta m}{\Delta t} $ \\ \hline
	Kraft via Impuls  & $\vec{F}_{res} = \vec{p}\,'$  \\ \hline
	Kraftstoss & $ \vec{F}\Delta t = \Delta\vec{p} $ & Falls $\vec{F}$ konst.\\ \hline
	Kraftstoff (Integral) & $ \int_{t_1}^{t_2} \vec{F}dt \,dx $ & Falls $\vec{F}$ variiert.\\ \hline
	Gleitreibungskraft & $ F_{GR} = \mu_G\cdot F_N $ \\ \hline
	Haftreibungskraft & $ F_{HR} = \mu_H\cdot F_N $ \\ \hline
	Rollwiderstandskraft & $ F_{RR} = \mu_R\cdot F_N $ \\ \hline
	Hangabtriebskraft & $ F_H = F_G \cdot sin(\alpha) $ & Schiefe Ebene \\ \hline
	Hangabtriebskraft & $ F_H = F_N \cdot tan(\alpha) $ & Schiefe Ebene \\ \hline
	Normalkraft & $ F_N = F_G \cdot cos(\alpha) $ & Schiefe Ebene \\ \hline
	Normalkraft & $ F_N = \frac{F_H}{tan(\alpha)} $ & Schiefe Ebene \\ \hline
	Steigung S & $S = \frac{h}{b}$ & auf b = 100m bezogen und mist in \% ausgedruckt \\ \hline
	Steig S - $cos(\alpha)$ & $cos(\alpha) = \frac{1}{\sqrt{S^2+1}}$ \\ \hline
	Steig S - $sin(\alpha)$ & $sin(\alpha) = \frac{\sqrt{S^2+1}}{S}$ \\ \hline
	Federkraft\linebreak (HOOKE'sches Gesetz) & $ F_F = D\cdot y $ & $y$ = $|l - l_o|$\linebreak $l_0$: entspannte Feder\linebreak l: gespannte Feder\linebreak $D$: Federkonstante $[D] = \frac{N}{m}$ \\ \hline
	Dichte & $\varrho = \frac{m}{V}$ & $[\varrho] = \frac{kg}{m^3}$ \\ \hline
	Strömungswiderstand & $ F_w = c_w \cdot \frac{\varrho v^2}{2}\cdot A $ & $c_w: Widerstandszahl$ \\ \hline
	Arbeit (Konstanter Kraft) & $ W = F \cdot \Delta s$ & $[W] = N\cdot m = W\cdot s = J$ \\ \hline
	Arbeit (Varriierte Kraft) & $ W=\int_{s_A}^{s_B} F_sds \,dx  $ \\ \hline
	Energie	& $E$ & $[E] = J = W\cdot s$\\ \hline
	Rel. Arbeit und Energie & $\Delta E = W_{AB}$ \\ \hline
	kin. Energie & $E_k = \frac{1}{2}mv^2$ \\ \hline
	pot. Energie & $E_p = mgh$ \\ \hline
	Spannenergie Feder & $E_f = \frac{1}{2}Dy^2$ \\ \hline
\end{tabularx}
\begin{tabularx}{\columnwidth}{@{}XXX@{}}
	Energieerhaltung            & $ \vec{E}_{tot} = \sum \vec{E}_i = konst $ & $\vec{E}_{tot}$: Gesamtenergie, $\vec{E}_{i}$: Teilenergie \\ \hline
	Energieerhaltungssatz (EES) & $\sum E_E = \sum E_A$                      & $E_{E1} + E_{E2} + ... = E_{A1} + E_{A2} + ...$            \\ \hline
	EES pot. \& kin. & $E_{pE} + E_{kinE} =$ & $E_{pA} + E_{kinA}$ \\ \hline
	Wirkungsgrad & $\mu = \frac{W_2}{W_1} = \frac{P_2}{P_1}$ & $W_1,P_1: aufgeno.\linebreak W_2,P_2: nutzbare$ \\ \hline
	Wirkungsgrad & zwischen mehrere Maschinen & $\mu_1 \cdot \mu_2 \cdot ...$ \\ \hline
	1 PS & $\approx 753.499 W$ \\ \hline
	Leistung & $P$ & $[P] = \frac{J}{s} = W\,(Watt)$ \\ \hline
	mittlere Leistung & $\overline{P} = \frac{W_{AB}}{\Delta t} = \frac{\Delta E}{\Delta t}$ \\ \hline
	momentane Leistung & $P = \frac{dW}{dt} = \vec{F}\cdot\vec{v}$ \\ \hline 
\end{tabularx}
\vspace{1mm}

\paragraph{Druck}\mbox{}\\
\begin{tabularx}{\columnwidth}{@{}XXX@{}}
	Druck & $p = \frac{F_N}{A}$ & $[p] = \frac{N}{m^2} = Pa$ \\ \hline                                                            
	Druck & $p = \frac{\Delta F_N}{\Delta A}$ & $[p] = \frac{N}{m^2} = Pa$ \\ \hline                                                            
	Schweredruck & $\Delta p = \varrho gh$ \\ \hline
	Barometrische Höhenformel & $p = p_0 \cdot e^{-\frac{\varrho_0g}{p_0}\cdot h}$ \\ \hline
\end{tabularx}