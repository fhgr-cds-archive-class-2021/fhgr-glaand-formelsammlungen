\noindent
\paragraph{Geschichtliche Entwicklung}\mbox{}\\
\textit{antike Naturphilosophie}\linebreak
\begin{itemize}
	\item Astronomische Beobachtungen der Ägypter, Babylonier und Naturphilosophie der Griechen - z.B. Aristoteles, Archimedes
	\item Entmythologisierung der Natur
	\item Akzeptierten jedoch nicht das Experiment als Prüfstein jeder Theorie - eine Naturbeobachtung genüge
\end{itemize}
\vspace{1mm}

\textit{Klassische Physik}\linebreak
\begin{itemize}
	\item Galilei erster Physiker im heutigen Sinne, da gezielte Experimente gemacht
	\item mit Hilfe der Mathematik Aufstellung von Gleichungen zur Erklärung der Beobachtungen
	\item Isaac Newton, Boyle und Mendelejev
\end{itemize}
\vspace{1mm}

\textit{Moderne Physik}\linebreak
\begin{itemize}
	\item Ende 19. jhh
	\item Quantentheorie, Relativitätstheorie, Teilchenphysik
\end{itemize}
\vspace{1mm}

\paragraph{SI-Basis-Einheiten}\mbox{}\\
\textit{Zeit}\linebreak
Die Sekunde, Symbol $s$, ist die SI-Einheit der Zeit.
Sie ist definiert durch den festen Zahlenwert für die Strahlungsfrequenz
des Caesium-Atoms $\Delta{v_{Cs}}$ von $9'192'631'770\frac{1}{s}$.
\vspace{1mm}

\textit{Länge}\linebreak
Der Meter, Symbol $m$, ist die SI-Einheit der Länge. Es ist definiert durch
den festen Zahlenwert für die Lichtgeschwindigkeit im Vakuum $c$ von
$299'792'458\frac{m}{s}$, wobei die Sekunde durch die Konstante $\Delta{v_{Cs}}$
definiert ist.
\vspace{1mm}

\textit{Masse}\linebreak
Das Kilogramm, Symbol $kg$, ist die SI-Einheit der Masse. Es ist definiert durch
den festen Zahlenwert für die PLANCK-Konstante $h$ von
$6,626'070'15 \cdot 10^{-34}\frac{kgm^2}{s}$, wobei der Meter und die Sekunde
durch die Konstanten $c$ und $\Delta{v_{Cs}}$ definiert sind.
\vspace{1mm}

\textit{Temperatur}\linebreak
Das Kelvin, Symbol $K$, ist die SI-Einheit der Temperatur.
Es ist definiert durch den Zahlenwert für die BOLTZMANN-Konstante $kB$
von $1,380'649\cdot10^{-23}\frac{kgm^2}{s^2K}$, wobei das Kilogramm, der Meter
und die Sekunde durch die Konstanten $h$, $c$ und $\Delta{v_{Cs}}$ definiert sind.
\vspace{1mm}

\textit{Stoffmenge}\linebreak
Die Stoffmenge, Symbol $n$, eines Systems ist ein Maß für die Anzahl spezifizierter
elementarer Einzelteile. Ein spezifiziertes elementares Einzelteil kann ein Atom,
ein Molekül, ein Ion, ein Elektron oder ein anderes Teilchen oder eine andere Gruppe
von Teilchen sein. Das Mol, Symbol $mol$, ist die SI-Einheit für die Stoffmenge. Ein Mol
enthält genau $6,022'140'76\cdot10^{23}$ spezifizierte elementare Einzelteile.
Diese Zahl ist der feste Zahlenwert für die AVOGADRO-Konstante $N_A$
von $6,022'140'76\cdot10^{23}\frac{1}{mol}$ und wird als AVOGADRO-Zahl bezeichnet.
\vspace{1mm}

\textit{Stromstärke}\linebreak
Das Ampere, Symbol $A$, ist die SI-Einheit der elektrischen Stromstärke. Sie ist definiert
durch den festen Zahlenwert für die Elementarladung $e$ von $1,602'176'634\cdot10^{-19}As$,
wobei die Sekunde durch die Konstante $\Delta{v_{Cs}}$ definiert ist.
\vspace{1mm}

\textit{Lichtstärke}\linebreak
Das Candela, Symbol $cd$, ist die SI-Einheit der Lichtstärke in einer vorgegebenen Richtung.
Es ist definiert durch den festen Zahlenwert für das photometrische Strahlungsäquivalent
$K_{cd}$ von $683\frac{cd\,sr\,s^3}{kg\,m^2}$, wobei das Kilogramm, der Meter und die Sekunde
durch die Konstanten $h$, $c$ und $\Delta{v_{Cs}}$ definiert sind.
\vspace{1mm}

\paragraph{Messfehler}\mbox{}\\
\begin{tabularx}{\columnwidth}{@{}X@{}}
	\hline
	- Systematisch (z.B. Messapparatur falsch kalibriert)                                            \\ \hline
	- Statistisch (Schwankungen um Mittelwert)                                                       \\ \hline
	- Genauigkeit über absoluten Fehler                                                              \\
	Bsp: $s = (4,3\pm 0,1) m$                                                                        \\
	$x$ sei exakter Wert                                                                             \\
	$\overline{x}$ sei Näherung für x                                                                \\ \hline
	absoluter Fehler: $\delta x= |x-\overline{x}| = |\overline{x}-x|$                                \\ \hline
	relativer Fehler: $\frac{\delta x}{|x|} \approx \frac{\delta x}{|x|} (Prozentangabe)$            \\ \hline
	Mittelwert: \[\overline{x} = \frac{1}{n} \sum_{i=1}^n x_i\]                                      \\ \hline
	Standardabweichung: \[ \Delta x = \sqrt{\frac{1}{n-1}\cdot \sum_{i=1}^n (x_i - \overline{x})} \] \\ \hline
	$[\overline{x}-\Delta x, \overline{x}+\Delta x]$ enthält ca. $\frac{2}{3}$ der Messwerte \\ \hline
	Absoluter FEhler einer Grösse ohne Fehlerangabe ist Einheit der letzten signifikaten Stelle
\end{tabularx}
\vspace{1mm}


\paragraph{Dezimalpräfixe}\mbox{}\\
\begin{tabularx}{\columnwidth}{@{}lXl|XlX@{}}
	\hline
	Symbol & Name  & Wert         & Symbol & Name  & Wert         \\ \hline
	$d$    & Dezi  & $ 10^{-1} 	$  & $da$   & Deka  & $ 10^{+1}  $ \\ \hline
	$c$    & Centi & $ 10^{-2} 	$  & $h$    & Hekto & $ 10^{+2}  $ \\ \hline
	$m$    & Milli & $ 10^{-3} 	$  & $k$    & Kilo  & $ 10^{+3}  $ \\ \hline
	$\mu$  & Mikro & $ 10^{-6} 	$  & $M$    & Mega  & $ 10^{+6}  $ \\ \hline
	$n$    & Nano  & $ 10^{-9} 	$  & $G$    & Giga  & $ 10^{+9}  $ \\ \hline
	$p$    & Piko  & $ 10^{-12} $ & $T$    & Tera  & $ 10^{+12} $ \\ \hline
	$f$    & Femto & $ 10^{-15} $ & $P$    & Peta  & $ 10^{+15} $ \\ \hline
	$a$    & Atto  & $ 10^{-18} $ & $E$    & Exa   & $ 10^{+18} $ \\ \hline
	$z$    & Zepto & $ 10^{-21} $ & $Z$    & Zetta & $ 10^{+21} $ \\ \hline
	$y$    & Yokto & $ 10^{-24} $ & $Y$    & Yotta & $ 10^{+24} $ \\ \hline
\end{tabularx}
\vspace{1mm}

\paragraph{Konstanten}\mbox{}\\
\begin{tabularx}{\columnwidth}{@{}lX@{}}
	\hline
	Name                      & Wert                                                               \\ \hline
	PI                        & $\pi \approx 3.141'592'653'589'793$                                \\
	Euler                     & $e \approx 2.718'281'828'459'045$                                  \\
	Lichtgeschwindigkeit      & $c \approx 299'792'458 \frac{m}{s}$                                \\
	Fallbeschleunigung        & $g \approx 9.81\frac{m}{s^2} $                                     \\
	AVOGADRO-Konstante        & $N_A = 6.022'140'76\cdot10^{23}\frac{1}{mol}$                      \\
	BOLTZMANN-Konstante       & $k_B = 1.380'649\cdot10^{-23}\frac{J}{K}$                          \\
	Universelle Gas-Konstante & $R = N_A\cdot k_B = 8.314'462 \frac{J}{\left( mol\cdot K \right)}$ \\
\end{tabularx}
\vspace{1mm}

\paragraph{Griechische Buchstaben}\mbox{}\\
\begin{tabularx}{\columnwidth}{@{}Xll|Xll@{}}
	\hline
	Buchstaben & klein                  & gross     & Buchstaben & klein          & gross    \\ \hline
	Alpha      & $\alpha$               & $A$       & Ny         & $\ni$          & $N$      \\ \hline
	Beta       & $\beta$                & $B$       & Xi         & $\xi$          & $\Xi$    \\ \hline
	Gamma      & $\gamma$               & $\Gamma$  & Omikron    & $o$            & $O$      \\ \hline
	Delta      & $\delta$               & $\Delta$  & Pi         & $\pi$          & $\Pi$    \\ \hline
	Epsilon    & $\epsilon,\varepsilon$ & $E$       & Rho        & $\rho$         & $P$      \\ \hline
	Zeta       & $\zeta$                & $Z$       & Sigma      & $\sigma$       & $\Sigma$ \\ \hline
	Eta        & $\eta$                 & $H$       & Tau        & $\tau$         & $T$      \\ \hline
	Theta      & $\theta,\vartheta$     & $\Theta$  & Ypsilon    & $\upsilon$     & $Y$      \\ \hline
	Iota       & $\iota$                & $I$       & Phi        & $\phi,\varphi$ & $\Phi$   \\ \hline
	Kappa      & $\kappa,\varkappa$     & $K$       & Chi        & $\chi$         & $X$      \\ \hline
	Lambda     & $\lambda$              & $\Lambda$ & Psi        & $\psi$         & $\Psi$   \\ \hline
	My         & $\mu$                  & $M$       & Omega      & $\omega$       & $\Omega$ \\ \hline
\end{tabularx}
\vspace{1mm}
